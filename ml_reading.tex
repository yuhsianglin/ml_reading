%%%%%%%% ICML 2021 EXAMPLE LATEX SUBMISSION FILE %%%%%%%%%%%%%%%%%

\documentclass{article}

% Recommended, but optional, packages for figures and better typesetting:
\usepackage{microtype}
\usepackage{graphicx}
\usepackage{subfigure}
\usepackage{booktabs} % for professional tables

% User-defined begins

\usepackage{makecell}
\usepackage{amsmath}
\usepackage{multirow}
\usepackage{xcolor}

\newcommand{\yhl}[1]{\textcolor{blue}{\bf\small [#1 --YHL]}}

% User-defined ends

% hyperref makes hyperlinks in the resulting PDF.
% If your build breaks (sometimes temporarily if a hyperlink spans a page)
% please comment out the following usepackage line and replace
% \usepackage{icml2021} with \usepackage[nohyperref]{icml2021} above.
\usepackage{hyperref}

% Attempt to make hyperref and algorithmic work together better:
\newcommand{\theHalgorithm}{\arabic{algorithm}}

% Use the following line for the initial blind version submitted for review:
%\usepackage{icml2021}

% If accepted, instead use the following line for the camera-ready submission:
\usepackage[accepted]{icml2021}

% The \icmltitle you define below is probably too long as a header.
% Therefore, a short form for the running title is supplied here:
\icmltitlerunning{Machine Learning Readings}

\begin{document}

\twocolumn[
\icmltitle{Machine Learning Readings}

% It is OKAY to include author information, even for blind
% submissions: the style file will automatically remove it for you
% unless you've provided the [accepted] option to the icml2021
% package.

% List of affiliations: The first argument should be a (short)
% identifier you will use later to specify author affiliations
% Academic affiliations should list Department, University, City, Region, Country
% Industry affiliations should list Company, City, Region, Country

% You can specify symbols, otherwise they are numbered in order.
% Ideally, you should not use this facility. Affiliations will be numbered
% in order of appearance and this is the preferred way.
\icmlsetsymbol{equal}{*}

\begin{icmlauthorlist}
\icmlauthor{Yu-Hsiang Lin}{}
\end{icmlauthorlist}

%\icmlaffiliation{institute}{Institute, City, State, USA}

\icmlcorrespondingauthor{Yu-Hsiang Lin}{yuhsianl@alumni.cmu.edu}

% You may provide any keywords that you
% find helpful for describing your paper; these are used to populate
% the "keywords" metadata in the PDF but will not be shown in the document
\icmlkeywords{Machine Learning}

\vskip 0.3in
]

% this must go after the closing bracket ] following \twocolumn[ ...

% This command actually creates the footnote in the first column
% listing the affiliations and the copyright notice.
% The command takes one argument, which is text to display at the start of the footnote.
% The \icmlEqualContribution command is standard text for equal contribution.
% Remove it (just {}) if you do not need this facility.

\printAffiliationsAndNotice{}  % leave blank if no need to mention equal contribution
%\printAffiliationsAndNotice{\icmlEqualContribution} % otherwise use the standard text.

\begin{abstract}
Papers and books in machine learning that I find interesting.
\end{abstract}







\section{Information Retrieval}

\subsection{Neural Ranking}

\subsubsection{Review}

\begin{enumerate}

\item An introduction \cite{mitra2018an}.

\item Actually a review of recommender systems \cite{Zhang2019}.

\item Are We Really Making Much Progress? A Worrying Analysis of Recent Neural Recommendation Approaches \cite{Dacrema2019}.

\end{enumerate}


\subsubsection{Models}

\begin{enumerate}

\item Deep Structured Semantic Model (DSSM) \cite{huang2013learning}.

\item Multiple document fields \cite{Zamani2018}.

\item Neural Factorization Machines \cite{He2017}.

\end{enumerate}




\section{Reinforcement Learning}

\subsection{General Review and Books}

\begin{enumerate}

\item Reinforcement Learning: An Introduction (2nd ed) \cite{Sutton2018}.

\item Deep Reinforcement Learning Hands-On \cite{Lapan2018}.

\end{enumerate}




\section{General Machine Learning}

\begin{enumerate}

\item The Deep Learning Book \cite{Goodfellow-et-al-2016}.

\item Machine Learning Yearning \cite{Ng2018}.

\item Machine Learning Engineering \cite{Burkov2020}.

\item (For general public) Probably Approximately Correct \cite{Valiant2013}.

\item The Hundred-Page Machine Learning Book \cite{Burkov2019}.

\item Dive into Deep Learning \cite{Zhang2021}.

\item (For general public) The Book of Why: The New Science of Cause and Effect \cite{Pearl2018}.

\end{enumerate}




\section{Haven't Got Time to Categorize}

\begin{enumerate}

\item The Deep Bootstrap Framework: Good Online Learners are Good Offline Generalizers \cite{nakkiran2021deep}.

\end{enumerate}




% \section*{Software and Data}
%
% If a paper is accepted, we strongly encourage the publication of software and data with the
% camera-ready version of the paper whenever appropriate. This can be
% done by including a URL in the camera-ready copy. However, \textbf{do not}
% include URLs that reveal your institution or identity in your
% submission for review. Instead, provide an anonymous URL or upload
% the material as ``Supplementary Material'' into the CMT reviewing
% system. Note that reviewers are not required to look at this material
% when writing their review.

% Acknowledgements should only appear in the accepted version.
% \section*{Acknowledgements}
%
% \textbf{Do not} include acknowledgements in the initial version of
% the paper submitted for blind review.
%
% If a paper is accepted, the final camera-ready version can (and
% probably should) include acknowledgements. In this case, please
% place such acknowledgements in an unnumbered section at the
% end of the paper. Typically, this will include thanks to reviewers
% who gave useful comments, to colleagues who contributed to the ideas,
% and to funding agencies and corporate sponsors that provided financial
% support.


% In the unusual situation where you want a paper to appear in the
% references without citing it in the main text, use \nocite
%\nocite{langley00}

\bibliography{ref_ml}
\bibliographystyle{icml2021}

\end{document}


% This document was modified from the file originally made available by
% Pat Langley and Andrea Danyluk for ICML-2K. This version was created
% by Iain Murray in 2018, and modified by Alexandre Bouchard in
% 2019 and 2021. Previous contributors include Dan Roy, Lise Getoor and Tobias
% Scheffer, which was slightly modified from the 2010 version by
% Thorsten Joachims & Johannes Fuernkranz, slightly modified from the
% 2009 version by Kiri Wagstaff and Sam Roweis's 2008 version, which is
% slightly modified from Prasad Tadepalli's 2007 version which is a
% lightly changed version of the previous year's version by Andrew
% Moore, which was in turn edited from those of Kristian Kersting and
% Codrina Lauth. Alex Smola contributed to the algorithmic style files.
